% Font size and paper size are specified
\documentclass[12pt,letterpaper]{article} 
%Set language to spanish
\usepackage[spanish, es-tabla]{babel}
% Specifies data for frontpage of document
\title{Práctica 2}
\author{Ayala Segoviano Donaldo Horacio}
\date{28 de Octubre del 2020}
% For margins
\usepackage{geometry}
\geometry{letterpaper, top=3cm, left=3cm, right=3cm}
% Numbers images according to sections
\usepackage{chngcntr}
\counterwithin{figure}{section}
% For images
\usepackage{graphicx}
%For formating links
\usepackage{hyperref}
\usepackage{natbib}
%Includes tools for making indexes
\usepackage{makeidx}
% Allows multiple columns
\usepackage{multicol}
%
\usepackage{fancyhdr}
\pagestyle{fancy}
\usepackage{amssymb}
\lhead{}
\rhead{Algoritmo de Thompson. Donaldo Ayala}

\usepackage{mathtools}
\DeclarePairedDelimiter\ceil{\lceil}{\rceil}
\DeclarePairedDelimiter\floor{\lfloor}{\rfloor}

\begin{document}


\centerline{\bf Compiladores, Sem: 2020-1, 3CV1, Práctica 2, 1 de Noviembre del 2020}
\centerline{}
\centerline{}
\begin{center}
\Large{\textsc{Práctica 2: Implementación del Algoritmo McNaughton-Yamada-Thompson}}
\end{center}
\centerline{}
\centerline{\bf {Ayala Segoviano Donaldo Horacio}}
\centerline{}
\centerline{Escuela Superior de Cómputo}
\centerline{Instituto Politécnico Nacional, México}
\centerline{$ayala.segoviano.donaldo@gmail.com$}
\newtheorem{Theorem}{\quad Theorem}[section]
\newtheorem{Definition}[Theorem]{\quad Definition}
\newtheorem{Corollary}[Theorem]{\quad Corollary}
\newtheorem{Lemma}[Theorem]{\quad Lemma}
\newtheorem{Example}[Theorem]{\quad Example}

\bigskip
\bigskip


\textbf{Objetivo:} Utilizando los principios de la programación Orientada a Objetos, diseñar y desarrollar la clase Thomsonconel método:\\
\begin{itemize}
    \item convertir(expresión\_regular:string): AFN
\end{itemize}

\bigskip
\bigskip

\setcounter{page}{1}
\thispagestyle{empty}



\section{Introducción}
El algoritmo AlgoritmoMcNaughton-Yamada-Thompson (también conocido como método de Thompson), sirve para obtener autómatas finitos no deterministas con transiciones vacías (AFND-$\epsilon$) a partir de expresiones regulares (ER).\\






\newpage
\section{Implementación de la clase Thompson}
\subsection{Clase Nodo}

\subsubsection{Descripción}
Para la implementación de la clase Thompson se implementó la clase Nodo, que permite generar arboles binarios de expresión y así simplificar la generación del AFN.\\

\subsubsection{Atributos}
La clase Nodo cuenta con los siguientes atributos:
\begin{itemize}
    \item (-)\textbf{Nodo*} hijo\_izquierdo : guarda el hijo izquierdo de dicho nodo del árbol.
    \item (-)\textbf{Nodo*} hijo\_derecho : guarda el hijo derecho de dicho nodo del árbol.
    \item (-)\textbf{char} símbolo : guarda el símbolo en la expresión regular.
\end{itemize}

\subsubsection{Métodos}
La clase Nodo cuenta con los siguientes métodos:
\begin{itemize}
    \item (+)\textbf{Nodo*} obtener\_hijo\_izquierdo(): regresa el hijo izquierdo del nodo.
    \item (+)\textbf{Nodo*} obtener\_hijo\_derecho(): regresa el hijo derecho del nodo.
    \item (+)\textbf{void} asignar\_hijo\_izquierdo (Nodo* nodo): asigna el parámetro nodo al hijo izquierdo.
    \item (+)\textbf{void} asignar\_hijo\_derecho (Nodo* nodo): asigna el parámetro nodo al hijo derecho.
    \item (+)\textbf{char} obtener\_simbolo : regresa el símbolo que contiene el nodo
    \item (+)\textbf{void} asignar\_simbolo(char sim) : asigna el símbolo sim al atributo símbolo. 
\end{itemize}

\subsection{Clase Thompson}

\subsubsection{Descripción}
La clase Thompson implementa el algoritmo de McNaughton-Yamada-Thompson que genera un AFN a partir de una expresión regular. \\

\subsubsection{Atributos}
La clase Nodo cuenta con los siguientes atributos:
\begin{itemize}
    \item (-)\textbf{Nodo*} arbol\_de\_expresion : es la raíz del árbol binario de expresión generado a partir de la expresión regular.
    \item (-)\textbf{$map<pair<int, char>,unordered_set<int>>$} transiciones : guarda las transiciones del autómata. el formato es el siguiente map$<$ {(estado\_actual, símbolo)}, estado\_siguiente $>$.
    
\end{itemize}

\subsubsection{Métodos}
La clase Nodo cuenta con los siguientes métodos:
\begin{itemize}
    \item (+)\textbf{AutomadaFinitoNoDeterminista} convertir( string expresión\_regular ): recibe una expresión regular y genera el AFN correspondiente mediante el método de Thompson.
    \item (-)\textbf{int} prioridad( char símbolo ): regresa la prioridad según la jerarquía de operadores en expresiones regulares.
    \item (-)\textbf{Nodo*} generar\_arbol\_de\_expresion( Nodo *arbol, string postfijo ): genera y devuelve la raíz a un árbol de expresión binario.
    \item (-)\textbf{void} inorden\_a\_postorden( string infijo, string \&postfijo ): Convierte la expresión regular de su forma infija a su forma postfija.
    \item (-)\textbf{bool} es\_caracter( char símbolo ): Regresa verdadero si el carácter enviado es una letra de [a-z], o falso en caso contrario.
    \item (-)\textbf{string} preprocesar\_infijo( string \&infijo ): añade el operador explícito '.' a la expresión regular.
    \item (-)\textbf{pair$<$ pair$<$int,int$>$, vector$<$pair$<$int,char$>> >$}: generar\_transiciones( Nodo* raiz, int \&id ): El método generar transiciones genera las transiciones del AFN a partir del árbol binario de de expresión de la expresión regular, regresa \{ (inicio, final), \{transiciones\_a\_final\}\} de la presente transición generada.
\end{itemize}


\newpage
\section{Conclusiones}
Al terminar la realización de la práctica se pudo llegar a la conclusión de que el algoritmo es de gran utilidad al momento de construir analizadores léxicos, ya que permite generar un autómata a partir de una expresión regular. En cuanto a la implementación, se llegó a la conclusión de que el tiempo de implementación sobrepasó el tiempo que se esperaba.

\end{document}